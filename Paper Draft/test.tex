\PassOptionsToPackage{unicode=true}{hyperref} % options for packages loaded elsewhere
\PassOptionsToPackage{hyphens}{url}
%
\documentclass[]{article}
\usepackage{lmodern}
\usepackage{amssymb,amsmath}
\usepackage{ifxetex,ifluatex}
\usepackage{fixltx2e} % provides \textsubscript
\ifnum 0\ifxetex 1\fi\ifluatex 1\fi=0 % if pdftex
  \usepackage[T1]{fontenc}
  \usepackage[utf8]{inputenc}
  \usepackage{textcomp} % provides euro and other symbols
\else % if luatex or xelatex
  \usepackage{unicode-math}
  \defaultfontfeatures{Ligatures=TeX,Scale=MatchLowercase}
\fi
% use upquote if available, for straight quotes in verbatim environments
\IfFileExists{upquote.sty}{\usepackage{upquote}}{}
% use microtype if available
\IfFileExists{microtype.sty}{%
\usepackage[]{microtype}
\UseMicrotypeSet[protrusion]{basicmath} % disable protrusion for tt fonts
}{}
\IfFileExists{parskip.sty}{%
\usepackage{parskip}
}{% else
\setlength{\parindent}{0pt}
\setlength{\parskip}{6pt plus 2pt minus 1pt}
}
\usepackage{hyperref}
\hypersetup{
            pdfborder={0 0 0},
            breaklinks=true}
\urlstyle{same}  % don't use monospace font for urls
\setlength{\emergencystretch}{3em}  % prevent overfull lines
\providecommand{\tightlist}{%
  \setlength{\itemsep}{0pt}\setlength{\parskip}{0pt}}
\setcounter{secnumdepth}{0}
% Redefines (sub)paragraphs to behave more like sections
\ifx\paragraph\undefined\else
\let\oldparagraph\paragraph
\renewcommand{\paragraph}[1]{\oldparagraph{#1}\mbox{}}
\fi
\ifx\subparagraph\undefined\else
\let\oldsubparagraph\subparagraph
\renewcommand{\subparagraph}[1]{\oldsubparagraph{#1}\mbox{}}
\fi

% set default figure placement to htbp
\makeatletter
\def\fps@figure{htbp}
\makeatother


\date{}

\begin{document}

\hypertarget{iron-oxide-exsolution-an-important-dynamo-power-source}{%
\section{Iron-Oxide Exsolution an Important Dynamo Power
Source}\label{iron-oxide-exsolution-an-important-dynamo-power-source}}

\hypertarget{abstract}{%
\subsection{Abstract}\label{abstract}}

Earth has been observed to have a global magnetic field for at least the
past 4.1 billion years {[}CITE{]}, but recently proposed high thermal
conductivity measurements {[}CITE{]} and young inner-core ages
{[}CITE{]} make it difficult to sustain Earth's magnetic field by
cooling and inner-core growth alone {[}CITE{]}. Exsolution of light
elements such as magnesium (O'Rourke and Stevenson 2016) and silicon
(Hirose et al.~2017) from Earth's core have been proposed as sources of
energy to resolve this problem. We show that exsolution of iron-oxide is
a powerful third source of energy and that the exsolution of all three
species is controlled by interactions with the mantle. We use coupled
whole-earth thermodynamic and chemical interaction model to estimate the
entropy available to power a dynamo across Earth's history and examine
signals and timing of inner-core nucleation.

\hypertarget{introduction}{%
\subsection{Introduction}\label{introduction}}

TODO

\hypertarget{model}{%
\subsection{Model}\label{model}}

We model the thermal evolution of the Earth using a coupled core-mantle
thermodynamic model following the methods of Stevenson (1983) for
parameterized mantle convection and Nimmo (2015) for core
thermodynamics. We add chemical reactions between the core and mantle to
this system, modeled as an interaction layer at the base of the mantle
that is allowed to chemically equilibriate with the core fluid. This
governs the exsolution of light elements from the core into the mantle,
which contribute to the heat and entropy budget of the core. This system
is cast as a system of ODEs and solved forward in time from Earth's
formation to the present day.

\hypertarget{thermodynamic-model}{%
\subsubsection{Thermodynamic Model}\label{thermodynamic-model}}

\hypertarget{core-thermodynamics}{%
\paragraph{Core Thermodynamics}\label{core-thermodynamics}}

We model the thermodynamics of the core by modifying the method of Nimmo
(2015). This model assumes the CMB heat flow is not significantly
subadiabatic, which is satisfied by all of our valid runs. Nimmo tracks
the contribution to the heat and entropy budgets of the core from all
relevant sources. To his formulation, we add terms for latent heat
release and gravitational energy release for MgO, SiO2, and FeO
exsolution from Earth's core. The gravitational energy and entropy
released by exsolution is expressed as

​ \$ Q\_\{g,i\} = \int \rho \psi \alpha\_i \frac{dc}{dt}dV\$,
\(E_{g,i} = Q_{g,i}/T_c\)

from Table 1 in that work. This requires an estimate of the
compositional expansion \(\alpha_i\) for each species. We use
\(\alpha_{MgO}\sim0.84\) (O'Rourke, Korenaga 2016) and
\(\alpha_{SiO2}\sim1.117\) (Hirose et al.~2017). For FeO, we perform a
simple hard-sphere estimate to compute the change in density between a
parcel of fluid with 99 wt\% Fe, 1 wt\% O and 100 wt\% Fe to obtain
\(\alpha_{FeO} \sim 0.28\).

We compute latent heat release for each species through
\(Q_{L,i} = L_{H,i}\frac{dm_i}{dt}\), where \(\frac{dm_i}{dt}\) is the
change in the mass of the species with time and \(L_{H,i}\) is the
latent heat of exsolution for each species. We use \(L_{H,SiO2} = 4300\)
kJ/kg (Hirose et al., 2017), while \$L\_\{H,MgO\} = L\_\{H,FeO\}=910 \$
kJ/kg represents a value intermediate between that for SiO2 and
inner-core solidifcation. This value has little effect on the system, as
the total heat released by exsolution of light elements is quite small
in all of our model runs. In fact, we find \textless{}5\% difference in
model run outcomes when setting all latent heat of exsolution values to
zero. Note also that unlike latent heat from inner-core growth, the
latent heat from exsolution does not contribute to the entropy
production in the core as exsolution occurs at the CMB. In fact,
exsolution has a negative effect on the energy available to power a
dynamo because it decreases the core cooling rate.

\hypertarget{mantle-thermodynamics}{%
\paragraph{Mantle Thermodynamics}\label{mantle-thermodynamics}}

Chemical exchange between the core and the interaction layer at the base
of the mantle cause only small changes to the physical properites of the
bulk medium and do not significantly affect mantle dynamics. Therefore,
we use a simple 1D parameterized whole-mantle convection model following
the method of Stevenson (1983). We use an Arrhenius mantle viscosity
with present day viscosities varying from \$\mu\_p=\(10\)\^{}\{19.5\}\$
to 10\(^{21}\) Pa s. We use Stevenson's values for all parameters except
for radiogenic heat production in the mantle, which we replace using the
method from Korenaga (2005) which includes four individual radioactive
species.

\hypertarget{chemical-reactions}{%
\subsubsection{Chemical Reactions}\label{chemical-reactions}}

\hypertarget{chemical-interaction-layer}{%
\paragraph{Chemical Interaction
Layer}\label{chemical-interaction-layer}}

We model chemical interactions between the core and mantle as a system
of equilibrium reactions with the base of the lower mantle.
First-principles calculations and seismic observations argue for a
pyrolitic lower mantle, necessitating the convesion of minerals such as
olivine and ferropericlase transported from the upper mantle into
Bridgemanite at depth. While the exact mechanisms of this process are
not well-understood, it requires interaction between mineral species at
relatively large distances over timescales of millions of years. Thus,
at long enough timescales the chemical state of the lower mantle should
be able to be modeled as an equilibrium reaction system.

A similar argument can be made about core-mantle interactions. Material
from the lower mantle can both diffuse into and exolve out of the core,
setting up a system of equlibrium reactions. To model this, we assume a
typical lengthscale for chemical interactions in the mantle and keep
track of the moles of each mantle mineral in this layer. There is
considerable uncertainty concerning the mechanisms by which Earth's core
and mantle interact, and therefore the appropriate lengthscale is
uncertain. If chemical interaction is dominated by chemical diffusion,
lengthscales of H\textasciitilde{}1-10m over 10Myr may be appropriate
{[}CITE{]}. If grain-boundary diffusion occurs at the CMB, core material
could interact H\textasciitilde{}100m {[}CITE{]}, while up to
H\textasciitilde{}1km could be appropriate if core material physically
intrudes into the lower mantle {[}CITE{]}.

Exsolved material from the core builds up in this layer over time,
altering the composition of the interaction layer and therefore
affecting the exsolution of core species. During chemical interactions,
this layer moves with the background mantle convection along the free
surface of the CMB, setting up a ``conveyer belt'' system as in figure
1. In this system, the layer is removed by mantle upwellings or plumes
and replaced with fresh background mantle due to downwellings or sinking
slabs on a timescale \(\tau\) set by the background mantle convection.
Analytical stability analyses and 2D numerical mantle convection codes
show that exsolved material in our simulations never reaches the
required thickness and buoyancy contrast to form Rayleigh-Taylor
instabilities, in contrast with other recent results (Helffrich 2018).

To model this ``conveyer belt'' layer overturn, we include a layer
erosion term in our system of differential equations that pushes each
species towards the background mantle composition value. Assuming a
pyrolitic background mantle composition and specifying a layer thickness
sets the total number of moles \$M\_\{b,i\} \$ of each species \(i\) in
the layer. Our equations are cast in terms of absolute number of moles,
so the erosion term \(\left(\frac{dM_i}{dt}\right)_{erosion}\) for each
species takes is a function only of the difference \(M_i - M_{b,i}\) and
the timescale of layer overturn \(\tau\) (see supplement).

In our model, we assume a present day \(\tau_p\) between 200 and 800
Myr. The early earth likely had more vigorous convection, so we set the
timescale at early earth \(\tau_0 = \tau_p/8\) and set \(\tau\) at
intermediate times using an exponential relationship with a timescale of
1 Byr.

We assume a pyrolitic background lower mantle based on seismic and
first-principles estimates. This consists of 83\% Bridgemanite, 16\%
ferroperilase, and 1\% free SiO2, with a magnesium number of 0.8
{[}CITE{]}.

\hypertarget{core-chemistry}{%
\paragraph{Core Chemistry}\label{core-chemistry}}

We include four components in the core in our model: Fe, Mg, Si, O. We
track the absolute number of moles of each species \(M_i\) in the core,
and calculate the mole fractions as \(X_i=M_i/M_c\) where \(M_c\) is the
total number of moles in the core. These species exchange with the
mantle through the equilibrium reactions ​
\[MgO (mantle) \leftrightarrow Mg^{(met)} + O^{(met)}\] ​
\[SiO_2 (mantle) \leftrightarrow Si^{(met)} +2 O^{(met)}\] ​
\[FeO (mantle) \leftrightarrow Fe^{(met)} +O^{(met)}\] as can be seen in
figure 1. Each of these reactions is governed by their temperature
dependent equilibrium constant. We use the values reported by Hirose et
al. (2017) for SiO2 and FeO and Badro et al. (2015) for MgO. For each
species we calculate the equilibrium constant as

​ \(K_{1} = a_{Mg}a_{O} / a_{MgO} = X_{Mg} X_O / X_{MgO}\)

assuming activities are all equal to one and other species present have
no influence. In our model, we only allow Mg, Si, and O to leave the
core, but not enter. An undersaturated core would allow light elements
from the mantle to dissolve into the core fluid at the CMB. However,
this would likely form a thin stratified layer at the CMB heavily
enriched in light elements, which would prevent the reaction from
continuing and would not mix into the bulk of the core. Thus, the bulk
of the core composition would remain unchanged.

We model the mantle interaction layer as including five species: MgO,
SiO\_2, FeO, MgSiO\_3, and FeSiO\_3. The total number of moles of each
species are tracked and governed by the reactions

\$ MgSiO\_3 \leftrightarrow MgO + SiO\_2\$ \$ FeSiO\_3
\leftrightarrow FeO + SiO\_2\$ \$ FeSiO\_3 + MgO \leftrightarrow FeO +
MgSiO\_3\$

where the third reaction maintains stoichiometry in the layer and has an
equilibrium constant defined as 1. This constrains K\_4 to equal K\_5.
The lower mantle is expected to be mostly Bridgemanite, so that K\_4 =
K\_5 \textless{}\textless{} 1.

\hypertarget{system-of-equations}{%
\subsubsection{System of Equations}\label{system-of-equations}}

To determine the evolution of these reaction equations over time, we
take the derivative of each equilibrium reaction with respect to
temperature. These equations are tightly coupled due to the dependence
of each mol fraction on both the number of moles of that species and the
total number of moles in the core or mantle. We use the open source
sympy linear equation solver to convert the equations into a system of
first-order nonlinear ODEs in order to integrate them into the
thermodynamic model.

For the fully coupled thermo-chemical model, we explicitly track the
temperatures of the upper mantle and core-mantle boundary, as well as
the absolute number of moles of each of the four species in the core and
five in the interaction layer. We specify the initial state by setting
\(T_{CMB,0}\) and \(T_{UM,0}\) as well as the initial composition of the
core \(M_{Fe}, M_{Mg}, M_{Si}, M_{O}\). This sets the initial
interaction layer composition by requiring it be in equilibrium with the
core at formation. We find that many initial core compositions require
initial interaction layer compositions very different from what we
expect at the present day. In particular, many initial compositions
require a layer highly enriched in ferropericlase. Thus, we adjust the
equlibrium constant governing the Bridgeanite to ferropericlase ratio
towards the expected present-day value on the same timescale \(\tau\) as
mantle overturn. Finally, we solve the complete system of equations
using scipy's built-in ODEint solver.

\hypertarget{results}{%
\subsection{Results}\label{results}}

Using this model, we compute the evolution of Earth for a range of
plausible parameters and keep solutions that result in inner-core sizes
within 10\% of the present-day value. We show the evolution of Earth for
one set of parameters in figure 2. In this run, entropy production is
initially dominated by secular cooling before FeO begins to exsolve out
of the core and becomes the dominant source of entropy production. Then,
as the core cools, both SiO2 and MgO begin to exsolve at distinct times,
causing small spikes in total entropy available but still contributing
less to the overall entropy budget than either FeO or secular cooling.
For the \textasciitilde{}2 Byrs before inner-core nucleation, secular
cooling is the largest single contributor to the entropy budget, but
exsolution of light elements stil produces a significant amount of
entropy. Inner-core nucleation brings a powerful new source of entropy
production which causes a spike in total available entropy for the
dynamo and depresses the core cooling rate. This causes entropy
production from cooling and exsolution to decline and allows inner-core
entropy production to dominate all other sources to the present day.

The mantle interaction layer is significantly enriched in iron in the
early days, with a Mg\# \textasciitilde{} 0.25 (see fig. 2d). As
exsolution slows near the present day, Mg is incorporated from
background mantle convection and the Mg\# increases to
\textasciitilde{}0.6. The layer is still enriched in iron at the present
day compared to the background mantle, which has a Mg\#
\textasciitilde{}0.9. This has possible implications for the origin of
LLSVPs and ULVZs, as many authors have proposed iron enrichment as an
origin of these features {[}TODO:CITE{]}. This plot represents a
``typical'' run of our model, but is far from the only possible Earth
history.

Table 1 \textbar{} Parameter \textbar{} Description \textbar{} Minimum
\textbar{} Maximum \textbar{} \textbar{} ----------------------
\textbar{} ----------- \textbar{} ----------- \textbar{} ---------
\textbar{} \textbar{} \(T_{CMB_0}\) \textbar{} initial CMB temperature
\textbar{} 5000K \textbar{} 6250K \textbar{} \textbar{} \(\nu_p\)
\textbar{} present-day mantle viscosity \textbar{} \(10^{19.5}\)
\textbar{} \(10^{21}\) \textbar{} \textbar{}\(w_{Mg,0}\)\textbar{}
initial Mg wt\% in core \textbar{} TODO \textbar{} \textbar{}
\textbar{}\(w_{Si,0}\)\textbar{} initial Si wt\% in core \textbar{}
\textbar{} \textbar{} \textbar{}\(w_{O,0}\)\textbar{} initial O wt\% in
core \textbar{} \textbar{} \textbar{} \textbar{} \(\tau_p\) \textbar{}
present-day mantle overturn timescale \textbar{} 200 Myrs \textbar{} 800
Myrs \textbar{} \textbar{} \(H\) \textbar{} interaction layer thickness
\textbar{} 30m \textbar{} 1000m \textbar{}

\hypertarget{entropy-production-over-time}{%
\subsubsection{Entropy Production Over
Time}\label{entropy-production-over-time}}

We run a suite of parameters, varying the values listed in table 1 and
keeping runs that have result in a present-day inner-core size within
10\% of the real value. We show entropy histories of all valid runs in
figure 3. Figure 3a shows the entropy available to power a dynamo,
demonstrating a huge variety of possible histories, but with consistent
trends between histories. There is generally a large amount of entropy
production in early Earth which declines to a minimum in the billion
years before inner-core nucleation before increasing suddenly as the
inner-core begins to grow. Some valid histories have only thermal
convection before inner-core nucleation, with no light-element
exsolution at all. However, most valid runs include a significant amount
of light-element exsolution.

FeO exsolution out of the core is a significant source of entropy
production in all computed histories that undergo light-element
exsolution. FeO contributes 50-100\% of the compostional entropy
production in the early earth, and 40-100\% just before inner-core
nucleation (figure 3b). In some histories MgO or SiO2 breifly produce
more entropy that FeO, but in almost all histories FeO is the largest
single source of compositional entropy production through the majority
of Earth's history. However, after the inner core begins to grow, FeO
exsolution becomes much less important, contributing less than 10\% of
the compositional entropy budget at the present day in most histories.

It is possible for cooling to either dominante or only produce a small
fraction of the available entropy before inner-core growth takes over.
However, general trends can be seen in figure 3c. Many histories show
that entropy production is initially dominated by cooling before one or
more elements begin to exsolve from the core and become the dominant
source of entropy. Ove time, cooling produces a steadily greater
fraction of total entropy until it contributes 30-75\% of total entropy
just before inner-core nucleation. Finally, after inner-core nucleation
takes over, it produces \textasciitilde{}15\% of total entropy at the
present day.

\hypertarget{inner-core-nucleation}{%
\subsubsection{Inner-Core Nucleation}\label{inner-core-nucleation}}

We use our suite of histories to examine the timinig of inner core
nucleation and the change in entropy production with inner-core
nucleation. In figure 4c we plot a histogram of inner-core nucleation
timing, showing that the median inner-core nucleation is
\textasciitilde{}550 Mya in our model, with nucleation ages varying from
\textasciitilde{}350Mya to \textasciitilde{}780Mya. These results are
very similar to those of Nimmo (2015), as might be expected since we use
a modified form of his core parameterization.

Our model always shows an increase in available entropy with inner-core
nucleation, with a median increase of \textasciitilde{}100\% from the
entropy available 1Myr before to 100Myr after nucleation. We show in
figure 3b the histogram of entropy changes ranging from
\textasciitilde{}5\% to \textasciitilde{}500\%. The largest entropy
increases arise due to histories with very low available entropy before
inner-core nucleation. Most entropy chanages in our model are much
higher than other recent estimates of the change in availble entropy
with inner-core nucleation (e.g.~Aubert 2017, CITE), but are again
consistent with Nimmo (2015). A doubling of availble entropy might
indicate that an increase in field stregth should be able to be observed
in the paleomagnetic record (CITE Driscol dynamo scaling). However,
available entropy might not correspond directly to a change in
observable field strength. Inner-core nucleation changes the location
and mechanism of the dominant source of entropy production, which could
change both the scaling of field strength with etropy and the mophology
of the generated field (CITE Nick's paper).

\hypertarget{conclusions}{%
\subsection{Conclusions}\label{conclusions}}

FeO exsolution is an important source of entropy to power Earth's
dynamo. Exsolution of light elements from Earth's core is governed by
the composition of the lowermost mantle. TODO

\end{document}
