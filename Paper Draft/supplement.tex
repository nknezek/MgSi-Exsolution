\documentclass[]{article}
\usepackage{amssymb,amsmath}
\usepackage{hyperref}
\usepackage{graphicx}

\date{}
\title{MgSi Paper Supplement}

\begin{document}

\maketitle

\section{Equilibrium Reactions}\label{equilibrium-reactions}

\begin{align}
K_{1} &= \frac{a_{Mg}a_{O}}{ a_{MgO}} = \frac{X_{Mg} X_O }{X_{MgO}} = \frac{M_{Mg}M_{O}M_m}{M_{MgO}M_c^2}\\
K_{2} &= \frac{a_{Si}a_{O}^2}{ a_{SiO_2}} = \frac{X_{Si} X_O^2 }{X_{SiO_2}} = \frac{M_{Si}M_{O}^2M_m}{M_{SiO_2}M_c^3}\\
K_{3} &= \frac{a_{Fe}a_{O}}{ a_{FeO}} = \frac{X_{Fe} X_O }{X_{FeO}} = \frac{M_{Fe}M_{O}M_m}{M_{FeO}M_c^2}\\
K_{4} &= \frac{a_{MgO}a_{SiO_2}}{ a_{MgSiO_3}} = \frac{X_{MgO} X_{SiO_2} }{X_{MgSiO_3}} = \frac{M_{MgO}M_{SiO_2}}{M_{MgSiO_3}M_m}​\\
K_{5} &= \frac{a_{FeO}a_{SiO_2}}{ a_{FeSiO_3}} = \frac{X_{FeO} X_{SiO_2} }{X_{FeSiO_3}} = \frac{M_{FeO}M_{SiO_2}}{M_{FeSiO_3}M_m}
\end{align}

\subsection{Change with Temperature}\label{change-with-temperature}
\begin{align}
\frac{(\partial_T K_1)}{K_1} &= \frac{(\partial_T M_{Mg})}{M_{Mg}} + \frac{(\partial_T M_{O})}{M_{O}} + \frac{(\partial_T M_{m})}{M_{m}} - \frac{2(\partial_T M_{c})}{M_{c}} - \frac{(\partial_T M_{MgO})}{M_{MgO}}
\\
\frac{1}{K_2}\partial_T K_2 &= \frac{1}{M_{Si}}\partial_T M_{Si} +
\frac{2}{M_{O}}\partial_T M_{O} + \frac{1}{M_{m}}\partial_T M_{m} -
\frac{3}{M_{c}}\partial_T M_{c} -
\frac{1}{M_{SiO_2}}\partial_T M_{SiO_2} \\
\frac{1}{K_3}\partial_T K_3 &= \frac{1}{M_{Fe}}\partial_T M_{Fe} +\frac{1}{M_{O}}\partial_T M_{O} +\frac{1}{M_{m}}\partial_T M_{m}-\frac{2}{M_{c}}\partial_T M_{c} - \frac{1}{M_{FeO}}\partial_T M_{FeO}\\
\frac{1}{K_4}\partial_T K_4 &= \frac{1}{M_{MgO}}\partial_T M_{MgO} +\frac{1}{M_{SiO_2}}\partial_T M_{SiO_2} -\frac{1}{M_{MgSiO_3}}\partial_T M_{MgSiO_3} -\frac{1}{M_{m}}\partial_T M_{m} \\
\frac{1}{K_5}\partial_T K_5 &= \frac{1}{M_{FeO}}\partial_T M_{FeO} +
\frac{1}{M_{SiO_2}}\partial_T M_{SiO_2} -
\frac{1}{M_{FeSiO_3}}\partial_T M_{FeSiO_3} -
\frac{1}{M_{m}}\partial_T M_{m} \\
\end{align}


\subsection{Molar Continuity}\label{molar-continuity}
\begin{align}
0 &= \partial_T M_{Mg} + \partial_T M_{MgO} + \partial_T M_{MgSiO_3} + \left(\partial_T M_{MgO}\right)_{erosion} + \left(\partial_T M_{MgSiO_3}\right)_{erosion} \\
0 &= \partial_T M_{Si} + \partial_T M_{SiO_2} + \partial_T M_{MgSiO_3} + \partial_T M_{FeSiO_3} + \left(\partial_T M_{SiO_2}\right)_{erosion} +\left(\partial_T M_{MgSiO_3}\right)_{erosion} +\left(\partial_T M_{FeSiO_3}\right)_{erosion}\\
0 &= \partial_T M_{Fe} + \partial_T M_{FeO} + \partial_T M_{FeSiO_3} + \left(\partial_T M_{FeO}\right)_{erosion} + \left(\partial_T M_{FeSiO_3}\right)_{erosion}\\
0 &= \partial_T M_{O}+\partial_T M_{MgO}+ \partial_T M_{FeO} +2\partial_T M_{SiO_2} + 3\partial_T M_{MgSiO_3} + 3\partial_T M_{FeSiO_3} +\left(\partial_T M_{MgO}\right)_{erosion} +\left(\partial_T M_{FeO}\right)_{erosion} +2\left(\partial_T M_{SiO2}\right)_{erosion} +3\left(\partial_T M_{MgSiO_3}\right)_{erosion} +3\left(\partial_T M_{FeSiO_3}\right)_{erosion}\\
0 &= \partial_T M_{m}+ \partial_T M_{MgO}+ \partial_T M_{FeO} +\partial_T M_{SiO_2} + \partial_T M_{MgSiO_3} + \partial_T M_{FeSiO_3} +\left(\partial_T M_{MgO}\right)_{erosion} +\left(\partial_T M_{FeO}\right)_{erosion} +\left(\partial_T M_{SiO2}\right)_{erosion} +\left(\partial_T M_{MgSiO_3}\right)_{erosion} +\left(\partial_T M_{FeSiO_3}\right)_{erosion}\\
0 &=\partial_T M_{c}+ \partial_T M_{Mg}+ \partial_T M_{Fe} +\partial_T M_{Si} + \partial_T M_{O} \\
\end{align}


\subsection{Erosion Term}\label{erosion-term}
\begin{equation}
\left(\partial_t M_i\right)_{erosion}=\frac{sgn(M_{b,i}-M_i)}{\tau}[(|M_i / M_{b,i}-1|+1)^2-1]
\end{equation}
\begin{equation}
\left(\partial_T M_i\right)_{erosion} = \left(\partial_t M_i\right)_{erosion} / \frac{T_c}{dt}
\end{equation}


\subsection{Solution} \label{solution}

With total moles in the mantle \(M_m\) and core \(M_c\), There are
eleven components \(M_i\) in the system and eleven equations. This
system is solved to obtain a system of first-order nonlinear ODEs of the
form \(\partial_T M_i = f(M_i, K_j)\) where \(M_i\) represent each
molar species in the mantle and core and \(K_j\) is each equilibrium
value. Due to the highly non-linear nature of this set of equations,
this results in the solution including several thousand terms each.
Further simplification of the equations may be possible, but there is no
significant performance penalty so it was not considered a problem. Note
that \(K_j =K_j(T_c)\)is an analytic function of temperature.


\section{Coupled Thermo-Chemical Evolution} \label{coupled-thermo-chemical-evolution}

The chemical reactions are incorporated into the core-mantle thermal
evolution model. With two temperatures \(T_{CMB}\) and \(T_{UM}\), and
nine molar species, this creates and eleven-component state vector
\(\mathbf{x}\). At each timestep, the mantle uses \(T_{CMB}\) and
\(T_{UM}\) to compute a heat flux at the surface and core-mantle
boundary and uses this to compute \(\partial_t T_{UM}\). Then, the
system of chemical reaction equations uses the current \(T_{CMB}\) and
molar concentrations \(M_i\) to compute \(\partial_T M_i\). The core
uses \(Q_{CMB}\) from the mantle and \(\partial_T M_i\) from exsolution
of light elements to compute \(\partial_t T_{CMB}\). Finally, this
change in temperature is used to compute the change in molar
concentrations with time by
\(\partial_t M_i = \partial_T M_i \partial_t T_{CMB}\). This allows
for the computation of \(\partial_t \mathbf{x}\) at any state.

We compute the heat release with change in CMB temperature using
\(\tilde{Q}_{g,i} \partial_t T_c = Q_{g,i}\) where \(\tilde{Q}_{g,i}\)
depends upon the change in wt\% of the light element with temperature
(see Nimmo 2015 eq. 73). This is easily computed given
\(\partial_T M_i\). Then, we include these in addition to the
inner-core and secular cooling terms to obtain
\(\tilde{Q}_T = \tilde{Q}_s+\tilde{Q}_g+\tilde{Q}_L +\tilde{Q}_{g,MgO}+\tilde{Q}_{g,FeO}+\tilde{Q}_{g,SiO2}+\tilde{Q}_{L,MgO}+\tilde{Q}_{L,FeO}+\tilde{Q}_{L,SiO2}\)

which is used the the calculation of the core energy budget (see Nimmo
2015 eq. 73).

In a similar manner, we modify the expression for entropy in the core by
including terms for exsolution of light elements
\(\tilde{E}_T = \tilde{E}_s+\tilde{E}_g+\tilde{E}_L+\tilde{E}_H +\tilde{E}_{g,MgO}+\tilde{E}_{g,FeO}+\tilde{E}_{g,SiO2}\)

(see Nimmo 2015, eq. 72). As noted in the main text, we do not allow
light elements to diffuse into the core because they would form a thin
stratified layer and fail to mix into the bulk core. We implement this
by requiring that \(\partial_t M_i \le 0\) for i=Mg,Si,O. We do allow
Fe to diffuse into the core, but this happens only rarely as a result of
changing mantle composition due to mantle overturn in our models and
must be matched by exsolution of light elements. Oxygen cannot diffuse
into the core and therefore the O concentration cannot increase, which
requires that any FeO entering the core be matched by MgO or SiO2
pulling Oxygen out at the same or faster rate.

\end{document}
